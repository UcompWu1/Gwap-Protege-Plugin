Section 5.1 timings: 

    relation suggestion: avg 36min
        21min  FJ
        40min  Stefan
        45min  Matyas 
        43min  Gerhard
        30min  Olga

    instanceof: avg 33.6min
        18min FJ
        13min olga
        Gerhard: 37min
        Philipp: 45min
        Matyas: 55min


description of relation detection experiment:

\textbf{Setup of the experiment}
To evaluate the plugin regarding the task \emph{Specification of Relation Type(T3)} we used two ontologies which were generated automatically
in the domains of climate change and tennis (the sport). These ontologies each contain 24 unnamed relations between subject and object concepts,
i.e. 48 unlabeled relations in total. Domain experts and crowd workers were asked to assign one of the relation types ``has-effect-on'', ``is-affected-by'', ``is-a'', ``is-opposite-of'', ``has-part'', ``is-part-of'', ``used-by'' or``uses'' to each unlabeled relation, or ``other'' is no type fits for the relation. 


\textbf{Results}
TODO

%%%%%%%%%%%%%%%%%%%%%%%%%%%%%%%%%%%%%%%%%%%%%%%%%%%%%%%%%%%%%%%%%%%%%%%%%%%%%%%%%%

\subsubsection{Feasibility Track}The input to most evaluation tasks are ontologies generated by the ontology learning algorithm described in~\cite{wohlgenannt2012} (primarily) from textual sources. %MS: I removed this text not because it is bad, but because it is not very relevant for the experiments and it might open us to some critique. Gerhard - if you really want it in, please include it again. For every ontology snapshot, ie. every result of an ontology learning stage, as well as the resulting ontology the ontology learning system exports an OWL file (using the Turtle seralization format, which can be converted to RDF/XML easily). The conversion of our internal format for lightweight ontologies is straightforward for concepts (which resemble to OWL classes). WordNet hyper- and hyponym relations are mapped to the OWL subClassOf property. The only challenging aspect is the representation of unlabeled relations in OWL. Our system creates ObjectProperties named "relation\_n", where "n" is an auto-incrementing number, as it is not possible to use the same ObjectProperty more than once. The label for those relations is plainly "relation". The periodically created versions of the domain ontology can be uploaded to a triple store and are thereby accessible via the SPARQL.
In the feasibility track, we evaluate the plugin over four ontologies covering four diverse domains (climate change, finance, tennis and wine).
All four domains are more or less general knowledge domains, but some (climate change, tennis) require domain familiarity or interest. 

\begin{table}
%\footnotesize
\center
\begin{tabular}{|l|c|c|c|c|c|} \hline
\textbf{Nr. of} &\textbf{Climate Change}&\textbf{Finance}&\textbf{Wine}&\textbf{Tennis}&\textbf{Human}\\
&\textbf{Ontology}&\textbf{Ontology}&\textbf{Ontology}&\textbf{Ontology}&\textbf{Ontology}\\\hline
\textbf{Classes} & 101 & 77 & 138 & 52 &3304\\ \hline
\textbf{Relations} & 77 & 50 & - & 67 & - \\ \hline
\textbf{IsA Relations} & 43 & 20 & 228 & 35 & 37\\ \hline
\textbf{Unnamed Relations} & 24 & 30 & - & 24 &-\\ \hline
\textbf{Instances} & 0 & 0 & 206& 0 & 0\\ \hline

\end{tabular}
\caption{Overview of the ontologies used in the feasibility and scalability evaluations.}
\label{table:ontology_data}
\end{table}

%@TBD (remove/change next sentence)
%More specialised domains will be evaluated as future research, but earlier work has already~\cite{Noy2013} investigated crowd-worker performance across ontologies
%of different domains/generality.
The ontologies tested as part of this evaluation experiments are of small to medium size (see Table~\ref{table:ontology_data}). The Climate Change ontology has 101 classes and 81 relations (out of which 43 are taxonomic relations, and 24 unnamed relations) while the Finance ontology has 77 classes and 50 relations (20 of which are taxonomic relations). The tennis ontology was used for the relation suggestion task only, it contains 24 unnamed relations. The ontologies were used as generated.

The ontologies used in the evaluation process, the instructions given to the manual evaluators, and the results, are found online\footnote{http://tinyurl.com/ucomp}. 
Additionally to automatically generated ontologies, we have made use of the Wine.owl ontology\footnote{\url{http://www.w3.org/TR/owl-guide/wine.rdf}} when evaluating the instanceOf verification tasks.
%Table~\ref{table:ontology_data} lists some ontology statistics, including the number of classes and subClassOf relations for the two ontologies.

